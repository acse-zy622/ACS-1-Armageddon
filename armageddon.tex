%% Generated by Sphinx.
\def\sphinxdocclass{report}
\documentclass[letterpaper,10pt,english]{sphinxmanual}
\ifdefined\pdfpxdimen
   \let\sphinxpxdimen\pdfpxdimen\else\newdimen\sphinxpxdimen
\fi \sphinxpxdimen=.75bp\relax
\ifdefined\pdfimageresolution
    \pdfimageresolution= \numexpr \dimexpr1in\relax/\sphinxpxdimen\relax
\fi
%% let collapsible pdf bookmarks panel have high depth per default
\PassOptionsToPackage{bookmarksdepth=5}{hyperref}


\PassOptionsToPackage{warn}{textcomp}
\usepackage[utf8]{inputenc}
\ifdefined\DeclareUnicodeCharacter
% support both utf8 and utf8x syntaxes
  \ifdefined\DeclareUnicodeCharacterAsOptional
    \def\sphinxDUC#1{\DeclareUnicodeCharacter{"#1}}
  \else
    \let\sphinxDUC\DeclareUnicodeCharacter
  \fi
  \sphinxDUC{00A0}{\nobreakspace}
  \sphinxDUC{2500}{\sphinxunichar{2500}}
  \sphinxDUC{2502}{\sphinxunichar{2502}}
  \sphinxDUC{2514}{\sphinxunichar{2514}}
  \sphinxDUC{251C}{\sphinxunichar{251C}}
  \sphinxDUC{2572}{\textbackslash}
\fi
\usepackage{cmap}
\usepackage[T1]{fontenc}
\usepackage{amsmath,amssymb,amstext}
\usepackage{babel}



\usepackage{tgtermes}
\usepackage{tgheros}
\renewcommand{\ttdefault}{txtt}



\usepackage[Bjarne]{fncychap}
\usepackage{sphinx}

\fvset{fontsize=auto}
\usepackage{geometry}


% Include hyperref last.
\usepackage{hyperref}
% Fix anchor placement for figures with captions.
\usepackage{hypcap}% it must be loaded after hyperref.
% Set up styles of URL: it should be placed after hyperref.
\urlstyle{same}


\usepackage{sphinxmessages}




\title{Armageddon}
\date{Nov 23, 2022}
\release{}
\author{unknown}
\newcommand{\sphinxlogo}{\vbox{}}
\renewcommand{\releasename}{}
\makeindex
\begin{document}

\ifdefined\shorthandoff
  \ifnum\catcode`\=\string=\active\shorthandoff{=}\fi
  \ifnum\catcode`\"=\active\shorthandoff{"}\fi
\fi

\pagestyle{empty}
\sphinxmaketitle
\pagestyle{plain}
\sphinxtableofcontents
\pagestyle{normal}
\phantomsection\label{\detokenize{index::doc}}



\chapter{Synopsis:}
\label{\detokenize{index:synopsis}}
\sphinxAtStartPar
Asteroids entering Earth’s atmosphere are subject to extreme drag forces
that decelerate, heat and disrupt the space rocks. The fate of an
asteroid is a complex function of its initial mass, speed, trajectory
angle and internal strength.

\sphinxAtStartPar
\sphinxhref{https://en.wikipedia.org/wiki/Asteroid}{Asteroids} 10\sphinxhyphen{}100 m in
diameter can penetrate deep into Earth’s atmosphere and disrupt
catastrophically, generating an atmospheric disturbance
(\sphinxhref{https://en.wikipedia.org/wiki/Air\_burst}{airburst}) that can cause
\sphinxhref{https://www.youtube.com/watch?v=tq02C\_3FvFo}{damage on the ground}.
Such an event occurred over the city of
\sphinxhref{https://en.wikipedia.org/wiki/Chelyabinsk\_meteor}{Chelyabinsk} in
Russia, in 2013, releasing energy equivalent to about 520 \sphinxhref{https://en.wikipedia.org/wiki/TNT\_equivalent}{kilotons of
TNT} (1 kt TNT is
equivalent to \(4.184 \times 10^{12}\) J), and injuring thousands of
people (\sphinxhref{http://doi.org/10.1126/science.1242642}{Popova et al.,
2013}; \sphinxhref{http://doi.org/10.1038/nature12741}{Brown et al.,
2013}). An even larger event
occurred over
\sphinxhref{https://en.wikipedia.org/wiki/Tunguska\_event}{Tunguska}, a
relatively unpopulated area in Siberia, in 1908.

\sphinxAtStartPar
This simulator predicts the fate of asteroids entering Earth’s atmosphere,
and provides a hazard mapper for an impact over the UK.


\chapter{Problem definition}
\label{\detokenize{index:problem-definition}}

\section{Equations of motion for a rigid asteroid}
\label{\detokenize{index:equations-of-motion-for-a-rigid-asteroid}}
\sphinxAtStartPar
The dynamics of an asteroid in Earth’s atmosphere prior to break\sphinxhyphen{}up is
governed by a coupled set of ordinary differential equations:
\begin{math}
\begin{aligned}
\frac{dv}{dt} & = \frac{-C_D\rho_a A v^2}{2 m} + g \sin \theta \\
\frac{dm}{dt} & = \frac{-C_H\rho_a A v^3}{2 Q} \\
\frac{d\theta}{dt} & = \frac{g\cos\theta}{v} - \frac{C_L\rho_a A v}{2 m} - \frac{v\cos\theta}{R_P + z} \\
\frac{dz}{dt} & = -v\sin\theta \\
\frac{dx}{dt} & = \frac{v\cos\theta}{1 + z/R_P}
\end{aligned}
\end{math}
\sphinxAtStartPar
In these equations, \(v\), \(m\), and \(A\) are the asteroid
speed (along trajectory), mass and cross\sphinxhyphen{}sectional area, respectively.
We will assume an initially \sphinxstylestrong{spherical asteroid} to convert from
inital radius to mass (and cross\sphinxhyphen{}sectional area). \(\theta\) is the
meteoroid trajectory angle to the horizontal (in radians), \(x\) is
the downrange distance of the meteoroid from its entry position,
\(z\) is the altitude and \(t\) is time; \(C_D\) is the drag
coefficient, \(\rho_a\) is the atmospheric density (a function of
altitude ), \(C_H\) is an ablation efficiency coefficient, \(Q\)
is the specific heat of ablation; \(C_L\) is a lift coefficient; and
\(R_P\) is the planetary radius. All terms use MKS units.


\section{Asteroid break\sphinxhyphen{}up and deformation}
\label{\detokenize{index:asteroid-break-up-and-deformation}}
\sphinxAtStartPar
A commonly used criterion for the break\sphinxhyphen{}up of an asteroid in the
atmosphere is when the ram pressure of the air interacting with the
asteroid \(\rho_a v^2\) first exceeds the strength of the asteroid
\(Y\).
\begin{equation*}
\begin{split}\rho_a v^2 = Y\end{split}
\end{equation*}
\sphinxAtStartPar
Should break\sphinxhyphen{}up occur, the asteroid deforms and spreads laterally as it
continues its passage through the atmosphere. Several models for the
spreading rate have been proposed. In the simplest model, the fragmented
asteroid’s spreading rate is related to its along trajectory speed
\sphinxhref{http://doi.org/10.1086/116499}{(Hills and Goda, 1993)}:
\begin{equation*}
\begin{split}\frac{dr}{dt} = \left[\frac{7}{2}\alpha\frac{\rho_a}{\rho_m}\right]^{1/2} v\end{split}
\end{equation*}
\sphinxAtStartPar
Where \(r\) is the asteroid radius, \(\rho_m\) is the asteroid
density (assumed constant) and \(\alpha\) is a spreading
coefficient, often taken to be 0.3. It is conventional to define the
cross\sphinxhyphen{}sectional area of the expanding cloud of fragments as
\(A = \pi r^2\) (i.e., assuming a circular cross\sphinxhyphen{}section), for use
in the above equations. Fragmentation and spreading \sphinxstylestrong{ceases} when the
ram pressure drops back below the strength of the meteoroid
\(\rho_a v^2 < Y\).


\section{Airblast damage}
\label{\detokenize{index:airblast-damage}}
\sphinxAtStartPar
The rapid deposition of energy in the atmosphere is analogous to an
explosion and so the environmental consequences of the airburst can be
estimated using empirical data from atmospheric explosion experiments
\sphinxhref{https://www.dtra.mil/Portals/61/Documents/NTPR/4-Rad\_Exp\_Rpts/36\_The\_Effects\_of\_Nuclear\_Weapons.pdf}{(Glasstone and Dolan,
1977)}.

\sphinxAtStartPar
The main cause of damage close to the impact site is a strong (pressure)
blastwave in the air, known as the \sphinxstylestrong{airblast}. Empirical data suggest
that the pressure in this wave \(p\) (in Pa) (above ambient, also
known as overpressure), as a function of explosion energy \(E_k\)
(in kilotons of TNT equivalent), burst altitude \(z_b\) (in m) and
horizontal range \(r\) (in m), is given by:
\begin{equation*}
   p(r) = 3.14 \times 10^{11} \left(\frac{r^2 + z_b^2}{E_k^{2/3}}\right)^{-1.3} + 1.8 \times 10^{7} \left(\frac{r^2 + z_b^2}{E_k^{2/3}}\right)^{-0.565}
\end{equation*}
\sphinxAtStartPar
For airbursts, we will take the total kinetic energy lost by the
asteroid at the burst altitude as the burst energy \(E_k\). For
cratering events, we will define \(E_k\)
as the \sphinxstylestrong{larger} of the total kinetic energy lost by the asteroid at
the burst altitude or the residual kinetic energy of the asteroid when
it hits the ground.

\sphinxAtStartPar
The following threshold pressures can then be used to define different
degrees of damage.


\begin{savenotes}\sphinxattablestart
\sphinxthistablewithglobalstyle
\centering
\begin{tabulary}{\linewidth}[t]{|T|T|T|}
\sphinxtoprule
\sphinxstyletheadfamily 
\sphinxAtStartPar
Damage Level
&\sphinxstyletheadfamily 
\sphinxAtStartPar
Description
&\sphinxstyletheadfamily 
\sphinxAtStartPar
Pressure (kPa)
\\
\sphinxmidrule
\sphinxtableatstartofbodyhook
\sphinxAtStartPar
1
&
\sphinxAtStartPar
\textasciitilde{}10\% glass windows shatter
&
\sphinxAtStartPar
1.0
\\
\sphinxhline
\sphinxAtStartPar
2
&
\sphinxAtStartPar
\textasciitilde{}90\% glass windows shatter
&
\sphinxAtStartPar
3.5
\\
\sphinxhline
\sphinxAtStartPar
3
&
\sphinxAtStartPar
Wood frame buildings collapse
&
\sphinxAtStartPar
27
\\
\sphinxhline
\sphinxAtStartPar
4
&
\sphinxAtStartPar
Multistory brick buildings collapse
&
\sphinxAtStartPar
43
\\
\sphinxbottomrule
\end{tabulary}
\sphinxtableafterendhook\par
\sphinxattableend\end{savenotes}

\sphinxAtStartPar
Table 1: Pressure thresholds (in kPa) for airblast damage


\section{Additional sections}
\label{\detokenize{index:additional-sections}}
\sphinxAtStartPar
You should expand this documentation to include explanatory text for all components of your tool.


\chapter{Function API}
\label{\detokenize{index:module-locator}}\label{\detokenize{index:function-api}}\index{module@\spxentry{module}!locator@\spxentry{locator}}\index{locator@\spxentry{locator}!module@\spxentry{module}}
\sphinxAtStartPar
Module dealing with postcode information.
\index{PostcodeLocator (class in locator)@\spxentry{PostcodeLocator}\spxextra{class in locator}}

\begin{fulllineitems}
\phantomsection\label{\detokenize{index:locator.PostcodeLocator}}
\pysigstartsignatures
\pysiglinewithargsret{\sphinxbfcode{\sphinxupquote{class\DUrole{w}{  }}}\sphinxcode{\sphinxupquote{locator.}}\sphinxbfcode{\sphinxupquote{PostcodeLocator}}}{\emph{\DUrole{n}{postcode\_file=\textquotesingle{}resources/full\_postcodes.csv\textquotesingle{}}}, \emph{\DUrole{n}{census\_file=\textquotesingle{}resources/population\_by\_postcode\_sector.csv\textquotesingle{}}}, \emph{\DUrole{n}{norm=\textless{}function great\_circle\_distance\textgreater{}}}}{}
\pysigstopsignatures
\sphinxAtStartPar
Class to interact with a postcode database file.
\begin{quote}\begin{description}
\sphinxlineitem{Parameters}\begin{itemize}
\item {} 
\sphinxAtStartPar
\sphinxstyleliteralstrong{\sphinxupquote{postcode\_file}} (\sphinxstyleliteralemphasis{\sphinxupquote{str}}\sphinxstyleliteralemphasis{\sphinxupquote{, }}\sphinxstyleliteralemphasis{\sphinxupquote{optional}}) \textendash{} Filename of a .csv file containing geographic
location data for postcodes.

\item {} 
\sphinxAtStartPar
\sphinxstyleliteralstrong{\sphinxupquote{census\_file}} (\sphinxstyleliteralemphasis{\sphinxupquote{str}}\sphinxstyleliteralemphasis{\sphinxupquote{, }}\sphinxstyleliteralemphasis{\sphinxupquote{optional}}) \textendash{} Filename of a .csv file containing census data by postcode sector.

\item {} 
\sphinxAtStartPar
\sphinxstyleliteralstrong{\sphinxupquote{norm}} (\sphinxstyleliteralemphasis{\sphinxupquote{function}}) \textendash{} Python function defining the distance between points in
latitude\sphinxhyphen{}longitude space.

\end{itemize}

\end{description}\end{quote}
\index{get\_population\_of\_postcode() (locator.PostcodeLocator method)@\spxentry{get\_population\_of\_postcode()}\spxextra{locator.PostcodeLocator method}}

\begin{fulllineitems}
\phantomsection\label{\detokenize{index:locator.PostcodeLocator.get_population_of_postcode}}
\pysigstartsignatures
\pysiglinewithargsret{\sphinxbfcode{\sphinxupquote{get\_population\_of\_postcode}}}{\emph{\DUrole{n}{postcodes}}, \emph{\DUrole{n}{sector}\DUrole{o}{=}\DUrole{default_value}{False}}}{}
\pysigstopsignatures
\sphinxAtStartPar
Return populations of a list of postcode units or sectors.
\begin{quote}\begin{description}
\sphinxlineitem{Parameters}\begin{itemize}
\item {} 
\sphinxAtStartPar
\sphinxstyleliteralstrong{\sphinxupquote{postcodes}} (\sphinxstyleliteralemphasis{\sphinxupquote{list}}\sphinxstyleliteralemphasis{\sphinxupquote{ of }}\sphinxstyleliteralemphasis{\sphinxupquote{lists}}) \textendash{} list of postcode units or postcode sectors

\item {} 
\sphinxAtStartPar
\sphinxstyleliteralstrong{\sphinxupquote{sector}} (\sphinxstyleliteralemphasis{\sphinxupquote{bool}}\sphinxstyleliteralemphasis{\sphinxupquote{, }}\sphinxstyleliteralemphasis{\sphinxupquote{optional}}) \textendash{} if true return populations for postcode sectors,
otherwise returns populations for postcode units

\end{itemize}

\sphinxlineitem{Returns}
\sphinxAtStartPar
Contains the populations of input postcode units or sectors

\sphinxlineitem{Return type}
\sphinxAtStartPar
list of lists

\end{description}\end{quote}
\subsubsection*{Examples}

\begin{sphinxVerbatim}[commandchars=\\\{\}]
\PYG{g+gp}{\PYGZgt{}\PYGZgt{}\PYGZgt{} }\PYG{n}{locator} \PYG{o}{=} \PYG{n}{PostcodeLocator}\PYG{p}{(}\PYG{l+s+s1}{\PYGZsq{}}\PYG{l+s+s1}{resources/full\PYGZus{}postcodes.csv}\PYG{l+s+s1}{\PYGZsq{}}\PYG{p}{,} \PYG{l+s+s1}{\PYGZsq{}}\PYG{l+s+s1}{resources/population\PYGZus{}by\PYGZus{}postcode\PYGZus{}sector.csv}\PYG{l+s+s1}{\PYGZsq{}}\PYG{p}{)}
\PYG{g+gp}{\PYGZgt{}\PYGZgt{}\PYGZgt{} }\PYG{n}{pop1} \PYG{o}{=} \PYG{n}{locator}\PYG{o}{.}\PYG{n}{get\PYGZus{}population\PYGZus{}of\PYGZus{}postcode}\PYG{p}{(}\PYG{p}{[}\PYG{p}{[}\PYG{l+s+s1}{\PYGZsq{}}\PYG{l+s+s1}{SW7 2AZ}\PYG{l+s+s1}{\PYGZsq{}}\PYG{p}{,} \PYG{l+s+s1}{\PYGZsq{}}\PYG{l+s+s1}{SW7 2BT}\PYG{l+s+s1}{\PYGZsq{}}\PYG{p}{,} \PYG{l+s+s1}{\PYGZsq{}}\PYG{l+s+s1}{SW7 2BU}\PYG{l+s+s1}{\PYGZsq{}}\PYG{p}{,} \PYG{l+s+s1}{\PYGZsq{}}\PYG{l+s+s1}{SW7 2DD}\PYG{l+s+s1}{\PYGZsq{}}\PYG{p}{]}\PYG{p}{]}\PYG{p}{)}
\PYG{g+gp}{\PYGZgt{}\PYGZgt{}\PYGZgt{} }\PYG{n}{pop1}
\PYG{g+go}{[[19, 19, 19, 19]]}
\PYG{g+gp}{\PYGZgt{}\PYGZgt{}\PYGZgt{} }\PYG{n}{pop2} \PYG{o}{=} \PYG{n}{locator}\PYG{o}{.}\PYG{n}{get\PYGZus{}population\PYGZus{}of\PYGZus{}postcode}\PYG{p}{(}\PYG{p}{[}\PYG{p}{[}\PYG{l+s+s1}{\PYGZsq{}}\PYG{l+s+s1}{SW7  2}\PYG{l+s+s1}{\PYGZsq{}}\PYG{p}{]}\PYG{p}{]}\PYG{p}{,} \PYG{k+kc}{True}\PYG{p}{)}
\PYG{g+gp}{\PYGZgt{}\PYGZgt{}\PYGZgt{} }\PYG{n}{pop2}
\PYG{g+go}{[[2283]]}
\end{sphinxVerbatim}

\end{fulllineitems}

\index{get\_postcodes\_by\_radius() (locator.PostcodeLocator method)@\spxentry{get\_postcodes\_by\_radius()}\spxextra{locator.PostcodeLocator method}}

\begin{fulllineitems}
\phantomsection\label{\detokenize{index:locator.PostcodeLocator.get_postcodes_by_radius}}
\pysigstartsignatures
\pysiglinewithargsret{\sphinxbfcode{\sphinxupquote{get\_postcodes\_by\_radius}}}{\emph{\DUrole{n}{X}}, \emph{\DUrole{n}{radii}}, \emph{\DUrole{n}{sector}\DUrole{o}{=}\DUrole{default_value}{False}}}{}
\pysigstopsignatures
\sphinxAtStartPar
Return (unit or sector) postcodes within specific distances of
input location.
\begin{quote}\begin{description}
\sphinxlineitem{Parameters}\begin{itemize}
\item {} 
\sphinxAtStartPar
\sphinxstyleliteralstrong{\sphinxupquote{X}} (\sphinxstyleliteralemphasis{\sphinxupquote{arraylike}}) \textendash{} Latitude\sphinxhyphen{}longitude pair of centre location

\item {} 
\sphinxAtStartPar
\sphinxstyleliteralstrong{\sphinxupquote{radii}} (\sphinxstyleliteralemphasis{\sphinxupquote{arraylike}}) \textendash{} array of radial distances from X

\item {} 
\sphinxAtStartPar
\sphinxstyleliteralstrong{\sphinxupquote{sector}} (\sphinxstyleliteralemphasis{\sphinxupquote{bool}}\sphinxstyleliteralemphasis{\sphinxupquote{, }}\sphinxstyleliteralemphasis{\sphinxupquote{optional}}) \textendash{} if true return postcode sectors, otherwise postcode units

\end{itemize}

\sphinxlineitem{Returns}
\sphinxAtStartPar
Contains the lists of postcodes closer than the elements
of radii to the location X.

\sphinxlineitem{Return type}
\sphinxAtStartPar
list of lists

\end{description}\end{quote}
\subsubsection*{Examples}

\begin{sphinxVerbatim}[commandchars=\\\{\}]
\PYG{g+gp}{\PYGZgt{}\PYGZgt{}\PYGZgt{} }\PYG{n}{locator} \PYG{o}{=} \PYG{n}{PostcodeLocator}\PYG{p}{(}\PYG{l+s+s1}{\PYGZsq{}}\PYG{l+s+s1}{resources/full\PYGZus{}postcodes.csv}\PYG{l+s+s1}{\PYGZsq{}}\PYG{p}{,} \PYG{l+s+s1}{\PYGZsq{}}\PYG{l+s+s1}{resources/population\PYGZus{}by\PYGZus{}postcode\PYGZus{}sector.csv}\PYG{l+s+s1}{\PYGZsq{}}\PYG{p}{)}
\PYG{g+gp}{\PYGZgt{}\PYGZgt{}\PYGZgt{} }\PYG{n}{postcodes} \PYG{o}{=} \PYG{n}{locator}\PYG{o}{.}\PYG{n}{get\PYGZus{}postcodes\PYGZus{}by\PYGZus{}radius}\PYG{p}{(}\PYG{p}{(}\PYG{l+m+mf}{51.4981}\PYG{p}{,} \PYG{o}{\PYGZhy{}}\PYG{l+m+mf}{0.1773}\PYG{p}{)}\PYG{p}{,} \PYG{p}{[}\PYG{l+m+mf}{0.13e3}\PYG{p}{]}\PYG{p}{)}
\PYG{g+gp}{\PYGZgt{}\PYGZgt{}\PYGZgt{} }\PYG{n}{postcode\PYGZus{}dictionaries} \PYG{o}{=} \PYG{p}{[}\PYG{n+nb}{dict}\PYG{o}{.}\PYG{n}{fromkeys}\PYG{p}{(}\PYG{n}{postcodes}\PYG{p}{[}\PYG{n}{i}\PYG{p}{]}\PYG{p}{,} \PYG{l+s+s2}{\PYGZdq{}}\PYG{l+s+s2}{risk}\PYG{l+s+s2}{\PYGZdq{}}\PYG{p}{)} \PYG{k}{for} \PYG{n}{i} \PYG{o+ow}{in} \PYG{n+nb}{range}\PYG{p}{(}\PYG{n+nb}{len}\PYG{p}{(}\PYG{n}{postcodes}\PYG{p}{)}\PYG{p}{)}\PYG{p}{]}
\PYG{g+gp}{\PYGZgt{}\PYGZgt{}\PYGZgt{} }\PYG{n}{ans1} \PYG{o}{=} \PYG{p}{[}\PYG{p}{\PYGZob{}}\PYG{l+s+s1}{\PYGZsq{}}\PYG{l+s+s1}{SW7 5HG}\PYG{l+s+s1}{\PYGZsq{}}\PYG{p}{:} \PYG{l+s+s1}{\PYGZsq{}}\PYG{l+s+s1}{risk}\PYG{l+s+s1}{\PYGZsq{}}\PYG{p}{,}\PYG{l+s+s1}{\PYGZsq{}}\PYG{l+s+s1}{SW7 2BU}\PYG{l+s+s1}{\PYGZsq{}}\PYG{p}{:} \PYG{l+s+s1}{\PYGZsq{}}\PYG{l+s+s1}{risk}\PYG{l+s+s1}{\PYGZsq{}}\PYG{p}{,}\PYG{l+s+s1}{\PYGZsq{}}\PYG{l+s+s1}{SW7 5HQ}\PYG{l+s+s1}{\PYGZsq{}}\PYG{p}{:} \PYG{l+s+s1}{\PYGZsq{}}\PYG{l+s+s1}{risk}\PYG{l+s+s1}{\PYGZsq{}}\PYG{p}{,}                    \PYG{l+s+s1}{\PYGZsq{}}\PYG{l+s+s1}{SW7 2BT}\PYG{l+s+s1}{\PYGZsq{}}\PYG{p}{:} \PYG{l+s+s1}{\PYGZsq{}}\PYG{l+s+s1}{risk}\PYG{l+s+s1}{\PYGZsq{}}\PYG{p}{,}\PYG{l+s+s1}{\PYGZsq{}}\PYG{l+s+s1}{SW7 5HF}\PYG{l+s+s1}{\PYGZsq{}}\PYG{p}{:} \PYG{l+s+s1}{\PYGZsq{}}\PYG{l+s+s1}{risk}\PYG{l+s+s1}{\PYGZsq{}}\PYG{p}{,}\PYG{l+s+s1}{\PYGZsq{}}\PYG{l+s+s1}{SW7 2DD}\PYG{l+s+s1}{\PYGZsq{}}\PYG{p}{:} \PYG{l+s+s1}{\PYGZsq{}}\PYG{l+s+s1}{risk}\PYG{l+s+s1}{\PYGZsq{}}\PYG{p}{,}                    \PYG{l+s+s1}{\PYGZsq{}}\PYG{l+s+s1}{SW7 2AZ}\PYG{l+s+s1}{\PYGZsq{}}\PYG{p}{:} \PYG{l+s+s1}{\PYGZsq{}}\PYG{l+s+s1}{risk}\PYG{l+s+s1}{\PYGZsq{}}\PYG{p}{\PYGZcb{}}\PYG{p}{]}
\PYG{g+gp}{\PYGZgt{}\PYGZgt{}\PYGZgt{} }\PYG{n}{postcode\PYGZus{}dictionaries} \PYG{o}{==} \PYG{n}{ans1}
\PYG{g+go}{True}
\PYG{g+gp}{\PYGZgt{}\PYGZgt{}\PYGZgt{} }\PYG{n}{postcodes} \PYG{o}{=} \PYG{n}{locator}\PYG{o}{.}\PYG{n}{get\PYGZus{}postcodes\PYGZus{}by\PYGZus{}radius}\PYG{p}{(}\PYG{p}{(}\PYG{l+m+mf}{51.4981}\PYG{p}{,} \PYG{o}{\PYGZhy{}}\PYG{l+m+mf}{0.1773}\PYG{p}{)}\PYG{p}{,} \PYG{p}{[}\PYG{l+m+mf}{0.4e3}\PYG{p}{,} \PYG{l+m+mf}{0.2e3}\PYG{p}{]}\PYG{p}{,} \PYG{k+kc}{True}\PYG{p}{)}
\PYG{g+gp}{\PYGZgt{}\PYGZgt{}\PYGZgt{} }\PYG{n}{postcode\PYGZus{}dictionaries} \PYG{o}{=} \PYG{p}{[}\PYG{n+nb}{dict}\PYG{o}{.}\PYG{n}{fromkeys}\PYG{p}{(}\PYG{n}{postcodes}\PYG{p}{[}\PYG{n}{i}\PYG{p}{]}\PYG{p}{,} \PYG{l+s+s2}{\PYGZdq{}}\PYG{l+s+s2}{risk}\PYG{l+s+s2}{\PYGZdq{}}\PYG{p}{)} \PYG{k}{for} \PYG{n}{i} \PYG{o+ow}{in} \PYG{n+nb}{range}\PYG{p}{(}\PYG{n+nb}{len}\PYG{p}{(}\PYG{n}{postcodes}\PYG{p}{)}\PYG{p}{)}\PYG{p}{]}
\PYG{g+gp}{\PYGZgt{}\PYGZgt{}\PYGZgt{} }\PYG{n}{ans2} \PYG{o}{=} \PYG{p}{[}\PYG{p}{\PYGZob{}}\PYG{l+s+s1}{\PYGZsq{}}\PYG{l+s+s1}{SW7 4}\PYG{l+s+s1}{\PYGZsq{}}\PYG{p}{:} \PYG{l+s+s1}{\PYGZsq{}}\PYG{l+s+s1}{risk}\PYG{l+s+s1}{\PYGZsq{}}\PYG{p}{,}\PYG{l+s+s1}{\PYGZsq{}}\PYG{l+s+s1}{SW7 5}\PYG{l+s+s1}{\PYGZsq{}}\PYG{p}{:} \PYG{l+s+s1}{\PYGZsq{}}\PYG{l+s+s1}{risk}\PYG{l+s+s1}{\PYGZsq{}}\PYG{p}{,}\PYG{l+s+s1}{\PYGZsq{}}\PYG{l+s+s1}{SW7 3}\PYG{l+s+s1}{\PYGZsq{}}\PYG{p}{:} \PYG{l+s+s1}{\PYGZsq{}}\PYG{l+s+s1}{risk}\PYG{l+s+s1}{\PYGZsq{}}\PYG{p}{,}                    \PYG{l+s+s1}{\PYGZsq{}}\PYG{l+s+s1}{SW7 1}\PYG{l+s+s1}{\PYGZsq{}}\PYG{p}{:} \PYG{l+s+s1}{\PYGZsq{}}\PYG{l+s+s1}{risk}\PYG{l+s+s1}{\PYGZsq{}}\PYG{p}{,}\PYG{l+s+s1}{\PYGZsq{}}\PYG{l+s+s1}{SW7 9}\PYG{l+s+s1}{\PYGZsq{}}\PYG{p}{:} \PYG{l+s+s1}{\PYGZsq{}}\PYG{l+s+s1}{risk}\PYG{l+s+s1}{\PYGZsq{}}\PYG{p}{,} \PYG{l+s+s1}{\PYGZsq{}}\PYG{l+s+s1}{SW7 2}\PYG{l+s+s1}{\PYGZsq{}}\PYG{p}{:} \PYG{l+s+s1}{\PYGZsq{}}\PYG{l+s+s1}{risk}\PYG{l+s+s1}{\PYGZsq{}}\PYG{p}{\PYGZcb{}}\PYG{p}{,}                   \PYG{p}{\PYGZob{}}\PYG{l+s+s1}{\PYGZsq{}}\PYG{l+s+s1}{SW7 4}\PYG{l+s+s1}{\PYGZsq{}}\PYG{p}{:} \PYG{l+s+s1}{\PYGZsq{}}\PYG{l+s+s1}{risk}\PYG{l+s+s1}{\PYGZsq{}}\PYG{p}{,}\PYG{l+s+s1}{\PYGZsq{}}\PYG{l+s+s1}{SW7 5}\PYG{l+s+s1}{\PYGZsq{}}\PYG{p}{:} \PYG{l+s+s1}{\PYGZsq{}}\PYG{l+s+s1}{risk}\PYG{l+s+s1}{\PYGZsq{}}\PYG{p}{,}\PYG{l+s+s1}{\PYGZsq{}}\PYG{l+s+s1}{SW7 3}\PYG{l+s+s1}{\PYGZsq{}}\PYG{p}{:} \PYG{l+s+s1}{\PYGZsq{}}\PYG{l+s+s1}{risk}\PYG{l+s+s1}{\PYGZsq{}}\PYG{p}{,}                    \PYG{l+s+s1}{\PYGZsq{}}\PYG{l+s+s1}{SW7 1}\PYG{l+s+s1}{\PYGZsq{}}\PYG{p}{:} \PYG{l+s+s1}{\PYGZsq{}}\PYG{l+s+s1}{risk}\PYG{l+s+s1}{\PYGZsq{}}\PYG{p}{,}\PYG{l+s+s1}{\PYGZsq{}}\PYG{l+s+s1}{SW7 9}\PYG{l+s+s1}{\PYGZsq{}}\PYG{p}{:} \PYG{l+s+s1}{\PYGZsq{}}\PYG{l+s+s1}{risk}\PYG{l+s+s1}{\PYGZsq{}}\PYG{p}{,}\PYG{l+s+s1}{\PYGZsq{}}\PYG{l+s+s1}{SW7 2}\PYG{l+s+s1}{\PYGZsq{}}\PYG{p}{:} \PYG{l+s+s1}{\PYGZsq{}}\PYG{l+s+s1}{risk}\PYG{l+s+s1}{\PYGZsq{}}\PYG{p}{\PYGZcb{}}\PYG{p}{]}
\PYG{g+gp}{\PYGZgt{}\PYGZgt{}\PYGZgt{} }\PYG{n}{postcode\PYGZus{}dictionaries} \PYG{o}{==} \PYG{n}{ans2}
\PYG{g+go}{True}
\end{sphinxVerbatim}

\end{fulllineitems}


\end{fulllineitems}

\index{great\_circle\_distance() (in module locator)@\spxentry{great\_circle\_distance()}\spxextra{in module locator}}

\begin{fulllineitems}
\phantomsection\label{\detokenize{index:locator.great_circle_distance}}
\pysigstartsignatures
\pysiglinewithargsret{\sphinxcode{\sphinxupquote{locator.}}\sphinxbfcode{\sphinxupquote{great\_circle\_distance}}}{\emph{\DUrole{n}{latlon1}}, \emph{\DUrole{n}{latlon2}}}{}
\pysigstopsignatures
\sphinxAtStartPar
Calculate the great circle distance (in metres) between pairs of
points specified as latitude and longitude on a spherical Earth
(with radius 6371 km).
\begin{quote}\begin{description}
\sphinxlineitem{Parameters}\begin{itemize}
\item {} 
\sphinxAtStartPar
\sphinxstyleliteralstrong{\sphinxupquote{latlon1}} (\sphinxstyleliteralemphasis{\sphinxupquote{arraylike}}) \textendash{} latitudes and longitudes of first point (as {[}n, 2{]} array for n points)

\item {} 
\sphinxAtStartPar
\sphinxstyleliteralstrong{\sphinxupquote{latlon2}} (\sphinxstyleliteralemphasis{\sphinxupquote{arraylike}}) \textendash{} latitudes and longitudes of second point (as {[}m, 2{]} array for m points)

\end{itemize}

\sphinxlineitem{Returns}
\sphinxAtStartPar
Distance in metres between each pair of points (as an n x m array)

\sphinxlineitem{Return type}
\sphinxAtStartPar
numpy.ndarray

\end{description}\end{quote}
\subsubsection*{Examples}

\begin{sphinxVerbatim}[commandchars=\\\{\}]
\PYG{g+gp}{\PYGZgt{}\PYGZgt{}\PYGZgt{} }\PYG{k+kn}{import} \PYG{n+nn}{numpy}
\PYG{g+gp}{\PYGZgt{}\PYGZgt{}\PYGZgt{} }\PYG{n}{fmt} \PYG{o}{=} \PYG{k}{lambda} \PYG{n}{x}\PYG{p}{:} \PYG{n}{numpy}\PYG{o}{.}\PYG{n}{format\PYGZus{}float\PYGZus{}scientific}\PYG{p}{(}\PYG{n}{x}\PYG{p}{,} \PYG{n}{precision}\PYG{o}{=}\PYG{l+m+mi}{3}\PYG{p}{)}
\PYG{g+gp}{\PYGZgt{}\PYGZgt{}\PYGZgt{} }\PYG{k}{with} \PYG{n}{numpy}\PYG{o}{.}\PYG{n}{printoptions}\PYG{p}{(}\PYG{n}{formatter}\PYG{o}{=}\PYG{p}{\PYGZob{}}\PYG{l+s+s1}{\PYGZsq{}}\PYG{l+s+s1}{all}\PYG{l+s+s1}{\PYGZsq{}}\PYG{p}{:} \PYG{n}{fmt}\PYG{p}{\PYGZcb{}}\PYG{p}{)}\PYG{p}{:}    \PYG{n+nb}{print}\PYG{p}{(}\PYG{n}{great\PYGZus{}circle\PYGZus{}distance}\PYG{p}{(}\PYG{p}{[}\PYG{p}{[}\PYG{l+m+mf}{54.0}\PYG{p}{,} \PYG{l+m+mf}{0.0}\PYG{p}{]}\PYG{p}{,} \PYG{p}{[}\PYG{l+m+mi}{55}\PYG{p}{,} \PYG{l+m+mf}{0.0}\PYG{p}{]}\PYG{p}{]}\PYG{p}{,} \PYG{p}{[}\PYG{l+m+mi}{55}\PYG{p}{,} \PYG{l+m+mf}{1.0}\PYG{p}{]}\PYG{p}{)}\PYG{p}{)}
\PYG{g+go}{[[1.286e+05]}
\PYG{g+go}{ [6.378e+04]]}
\end{sphinxVerbatim}

\end{fulllineitems}

\phantomsection\label{\detokenize{index:module-solver}}\index{module@\spxentry{module}!solver@\spxentry{solver}}\index{solver@\spxentry{solver}!module@\spxentry{module}}\index{Planet (class in solver)@\spxentry{Planet}\spxextra{class in solver}}

\begin{fulllineitems}
\phantomsection\label{\detokenize{index:solver.Planet}}
\pysigstartsignatures
\pysiglinewithargsret{\sphinxbfcode{\sphinxupquote{class\DUrole{w}{  }}}\sphinxcode{\sphinxupquote{solver.}}\sphinxbfcode{\sphinxupquote{Planet}}}{\emph{\DUrole{n}{atmos\_func}\DUrole{o}{=}\DUrole{default_value}{\textquotesingle{}exponential\textquotesingle{}}}, \emph{\DUrole{n}{atmos\_filename}\DUrole{o}{=}\DUrole{default_value}{\textquotesingle{}/Users/wz22/acs\sphinxhyphen{}armageddon\sphinxhyphen{}Dimorphos/armageddon/../resources/AltitudeDensityTable.csv\textquotesingle{}}}, \emph{\DUrole{n}{Cd}\DUrole{o}{=}\DUrole{default_value}{1.0}}, \emph{\DUrole{n}{Ch}\DUrole{o}{=}\DUrole{default_value}{0.1}}, \emph{\DUrole{n}{Q}\DUrole{o}{=}\DUrole{default_value}{10000000.0}}, \emph{\DUrole{n}{Cl}\DUrole{o}{=}\DUrole{default_value}{0.001}}, \emph{\DUrole{n}{alpha}\DUrole{o}{=}\DUrole{default_value}{0.3}}, \emph{\DUrole{n}{Rp}\DUrole{o}{=}\DUrole{default_value}{6371000.0}}, \emph{\DUrole{n}{g}\DUrole{o}{=}\DUrole{default_value}{9.81}}, \emph{\DUrole{n}{H}\DUrole{o}{=}\DUrole{default_value}{8000.0}}, \emph{\DUrole{n}{rho0}\DUrole{o}{=}\DUrole{default_value}{1.2}}}{}
\pysigstopsignatures
\sphinxAtStartPar
The class called Planet is initialised with constants appropriate
for the given target planet, including the atmospheric density profile
and other constants

\sphinxAtStartPar
Set up the initial parameters and constants for the target planet
\begin{quote}\begin{description}
\sphinxlineitem{Parameters}\begin{itemize}
\item {} 
\sphinxAtStartPar
\sphinxstyleliteralstrong{\sphinxupquote{atmos\_func}} (\sphinxstyleliteralemphasis{\sphinxupquote{string}}\sphinxstyleliteralemphasis{\sphinxupquote{, }}\sphinxstyleliteralemphasis{\sphinxupquote{optional}}) \textendash{} Function which computes atmospheric density, rho, at altitude, z.
Default is the exponential function rho = rho0 exp(\sphinxhyphen{}z/H).
Options are ‘exponential’, ‘tabular’ and ‘constant’

\item {} 
\sphinxAtStartPar
\sphinxstyleliteralstrong{\sphinxupquote{atmos\_filename}} (\sphinxstyleliteralemphasis{\sphinxupquote{string}}\sphinxstyleliteralemphasis{\sphinxupquote{, }}\sphinxstyleliteralemphasis{\sphinxupquote{optional}}) \textendash{} Name of the filename to use with the tabular atmos\_func option

\item {} 
\sphinxAtStartPar
\sphinxstyleliteralstrong{\sphinxupquote{Cd}} (\sphinxstyleliteralemphasis{\sphinxupquote{float}}\sphinxstyleliteralemphasis{\sphinxupquote{, }}\sphinxstyleliteralemphasis{\sphinxupquote{optional}}) \textendash{} The drag coefficient

\item {} 
\sphinxAtStartPar
\sphinxstyleliteralstrong{\sphinxupquote{Ch}} (\sphinxstyleliteralemphasis{\sphinxupquote{float}}\sphinxstyleliteralemphasis{\sphinxupquote{, }}\sphinxstyleliteralemphasis{\sphinxupquote{optional}}) \textendash{} The heat transfer coefficient

\item {} 
\sphinxAtStartPar
\sphinxstyleliteralstrong{\sphinxupquote{Q}} (\sphinxstyleliteralemphasis{\sphinxupquote{float}}\sphinxstyleliteralemphasis{\sphinxupquote{, }}\sphinxstyleliteralemphasis{\sphinxupquote{optional}}) \textendash{} The heat of ablation (J/kg)

\item {} 
\sphinxAtStartPar
\sphinxstyleliteralstrong{\sphinxupquote{Cl}} (\sphinxstyleliteralemphasis{\sphinxupquote{float}}\sphinxstyleliteralemphasis{\sphinxupquote{, }}\sphinxstyleliteralemphasis{\sphinxupquote{optional}}) \textendash{} Lift coefficient

\item {} 
\sphinxAtStartPar
\sphinxstyleliteralstrong{\sphinxupquote{alpha}} (\sphinxstyleliteralemphasis{\sphinxupquote{float}}\sphinxstyleliteralemphasis{\sphinxupquote{, }}\sphinxstyleliteralemphasis{\sphinxupquote{optional}}) \textendash{} Dispersion coefficient

\item {} 
\sphinxAtStartPar
\sphinxstyleliteralstrong{\sphinxupquote{Rp}} (\sphinxstyleliteralemphasis{\sphinxupquote{float}}\sphinxstyleliteralemphasis{\sphinxupquote{, }}\sphinxstyleliteralemphasis{\sphinxupquote{optional}}) \textendash{} Planet radius (m)

\item {} 
\sphinxAtStartPar
\sphinxstyleliteralstrong{\sphinxupquote{rho0}} (\sphinxstyleliteralemphasis{\sphinxupquote{float}}\sphinxstyleliteralemphasis{\sphinxupquote{, }}\sphinxstyleliteralemphasis{\sphinxupquote{optional}}) \textendash{} Air density at zero altitude (kg/m\textasciicircum{}3)

\item {} 
\sphinxAtStartPar
\sphinxstyleliteralstrong{\sphinxupquote{g}} (\sphinxstyleliteralemphasis{\sphinxupquote{float}}\sphinxstyleliteralemphasis{\sphinxupquote{, }}\sphinxstyleliteralemphasis{\sphinxupquote{optional}}) \textendash{} Surface gravity (m/s\textasciicircum{}2)

\item {} 
\sphinxAtStartPar
\sphinxstyleliteralstrong{\sphinxupquote{H}} (\sphinxstyleliteralemphasis{\sphinxupquote{float}}\sphinxstyleliteralemphasis{\sphinxupquote{, }}\sphinxstyleliteralemphasis{\sphinxupquote{optional}}) \textendash{} Atmospheric scale height (m)

\end{itemize}

\end{description}\end{quote}
\index{RK4\_helper() (solver.Planet method)@\spxentry{RK4\_helper()}\spxextra{solver.Planet method}}

\begin{fulllineitems}
\phantomsection\label{\detokenize{index:solver.Planet.RK4_helper}}
\pysigstartsignatures
\pysiglinewithargsret{\sphinxbfcode{\sphinxupquote{RK4\_helper}}}{\emph{\DUrole{n}{timestep}}}{}
\pysigstopsignatures
\sphinxAtStartPar
Helper function for RK4 method
\begin{quote}\begin{description}
\sphinxlineitem{Parameters}
\sphinxAtStartPar
\sphinxstyleliteralstrong{\sphinxupquote{timestep}} (\sphinxstyleliteralemphasis{\sphinxupquote{float}}) \textendash{} The stepsize of iteration

\sphinxlineitem{Returns}
\sphinxAtStartPar
\sphinxstylestrong{change} \textendash{} A numpy array containing the change of each variable.
Includes the following variables:
‘angle’, ‘radius’, ‘altitude’,
‘velocity’, ‘mass’, ‘distance’

\sphinxlineitem{Return type}
\sphinxAtStartPar
ndarray

\end{description}\end{quote}

\end{fulllineitems}

\index{analyse\_outcome() (solver.Planet method)@\spxentry{analyse\_outcome()}\spxextra{solver.Planet method}}

\begin{fulllineitems}
\phantomsection\label{\detokenize{index:solver.Planet.analyse_outcome}}
\pysigstartsignatures
\pysiglinewithargsret{\sphinxbfcode{\sphinxupquote{analyse\_outcome}}}{\emph{\DUrole{n}{result}}}{}
\pysigstopsignatures
\sphinxAtStartPar
Inspect a pre\sphinxhyphen{}found solution to calculate the impact and airburst stats
\begin{quote}\begin{description}
\sphinxlineitem{Parameters}
\sphinxAtStartPar
\sphinxstyleliteralstrong{\sphinxupquote{result}} (\sphinxstyleliteralemphasis{\sphinxupquote{DataFrame}}) \textendash{} pandas dataframe with velocity, mass, angle, altitude, horizontal
distance, radius and dedz as a function of time

\sphinxlineitem{Returns}
\sphinxAtStartPar

\sphinxAtStartPar
\sphinxstylestrong{outcome} \textendash{} dictionary with details of the impact event, which should contain
the key:
\begin{quote}

\sphinxAtStartPar
\sphinxcode{\sphinxupquote{outcome}} (which should contain one of the
following strings: \sphinxcode{\sphinxupquote{Airburst}} or \sphinxcode{\sphinxupquote{Cratering}}),
\end{quote}
\begin{description}
\sphinxlineitem{as well as the following 4 keys:}
\sphinxAtStartPar
\sphinxcode{\sphinxupquote{burst\_peak\_dedz}}, \sphinxcode{\sphinxupquote{burst\_altitude}},
\sphinxcode{\sphinxupquote{burst\_distance}}, \sphinxcode{\sphinxupquote{burst\_energy}}

\end{description}


\sphinxlineitem{Return type}
\sphinxAtStartPar
Dict

\end{description}\end{quote}

\end{fulllineitems}

\index{calculate\_energy() (solver.Planet method)@\spxentry{calculate\_energy()}\spxextra{solver.Planet method}}

\begin{fulllineitems}
\phantomsection\label{\detokenize{index:solver.Planet.calculate_energy}}
\pysigstartsignatures
\pysiglinewithargsret{\sphinxbfcode{\sphinxupquote{calculate\_energy}}}{\emph{\DUrole{n}{result}}}{}
\pysigstopsignatures
\sphinxAtStartPar
Function to calculate the kinetic energy lost per unit altitude in
kilotons TNT per km, for a given solution.
\begin{quote}\begin{description}
\sphinxlineitem{Parameters}\begin{itemize}
\item {} 
\sphinxAtStartPar
\sphinxstyleliteralstrong{\sphinxupquote{result}} (\sphinxstyleliteralemphasis{\sphinxupquote{DataFrame}}) \textendash{} A pandas dataframe with columns for the velocity, mass, angle,
altitude, horizontal distance and radius as a function of time

\item {} 
\sphinxAtStartPar
\sphinxstyleliteralstrong{\sphinxupquote{Returns}} (\sphinxstyleliteralemphasis{\sphinxupquote{DataFrame}}) \textendash{} Returns the dataframe with additional column \sphinxcode{\sphinxupquote{dedz}} which is the
kinetic energy lost per unit altitude

\end{itemize}

\end{description}\end{quote}

\end{fulllineitems}

\index{calculator\_rk4() (solver.Planet method)@\spxentry{calculator\_rk4()}\spxextra{solver.Planet method}}

\begin{fulllineitems}
\phantomsection\label{\detokenize{index:solver.Planet.calculator_rk4}}
\pysigstartsignatures
\pysiglinewithargsret{\sphinxbfcode{\sphinxupquote{calculator\_rk4}}}{\emph{\DUrole{n}{variables}}}{}
\pysigstopsignatures
\sphinxAtStartPar
Calculate the change of variables at given point
\begin{quote}\begin{description}
\sphinxlineitem{Parameters}
\sphinxAtStartPar
\sphinxstyleliteralstrong{\sphinxupquote{variables}} (\sphinxstyleliteralemphasis{\sphinxupquote{float}}) \textendash{} Angle, radius, altitude, velocity, mass, distance at currenty step

\sphinxlineitem{Returns}
\sphinxAtStartPar
\sphinxstylestrong{result} \textendash{} A numpy array containing the change of each variable.
Includes the following variables:
‘angle’, ‘radius’, ‘altitude’,
‘velocity’, ‘mass’, ‘distance’

\sphinxlineitem{Return type}
\sphinxAtStartPar
ndarray

\end{description}\end{quote}

\end{fulllineitems}

\index{create\_tabular\_density() (solver.Planet method)@\spxentry{create\_tabular\_density()}\spxextra{solver.Planet method}}

\begin{fulllineitems}
\phantomsection\label{\detokenize{index:solver.Planet.create_tabular_density}}
\pysigstartsignatures
\pysiglinewithargsret{\sphinxbfcode{\sphinxupquote{create\_tabular\_density}}}{\emph{\DUrole{n}{filename}\DUrole{o}{=}\DUrole{default_value}{\textquotesingle{}./resources/AltitudeDensityTable.csv\textquotesingle{}}}}{}
\pysigstopsignatures
\sphinxAtStartPar
Create a function given altitude return the density of atomosphere
using tabulated value
\begin{quote}\begin{description}
\sphinxlineitem{Parameters}
\sphinxAtStartPar
\sphinxstyleliteralstrong{\sphinxupquote{filename}} (\sphinxstyleliteralemphasis{\sphinxupquote{str}}\sphinxstyleliteralemphasis{\sphinxupquote{, }}\sphinxstyleliteralemphasis{\sphinxupquote{optional}}) \textendash{} Path to the tabular. default=”./resources/AltitudeDensityTable.csv”

\sphinxlineitem{Returns}
\sphinxAtStartPar
\sphinxstylestrong{tabular\_density} \textendash{} A function that takes altitude as input and return the density of
atomosphere density at given altitude.

\sphinxlineitem{Return type}
\sphinxAtStartPar
function

\end{description}\end{quote}

\end{fulllineitems}

\index{solve\_atmospheric\_entry() (solver.Planet method)@\spxentry{solve\_atmospheric\_entry()}\spxextra{solver.Planet method}}

\begin{fulllineitems}
\phantomsection\label{\detokenize{index:solver.Planet.solve_atmospheric_entry}}
\pysigstartsignatures
\pysiglinewithargsret{\sphinxbfcode{\sphinxupquote{solve\_atmospheric\_entry}}}{\emph{\DUrole{n}{radius}}, \emph{\DUrole{n}{velocity}}, \emph{\DUrole{n}{density}}, \emph{\DUrole{n}{strength}}, \emph{\DUrole{n}{angle}}, \emph{\DUrole{n}{init\_altitude}\DUrole{o}{=}\DUrole{default_value}{100000.0}}, \emph{\DUrole{n}{dt}\DUrole{o}{=}\DUrole{default_value}{0.05}}, \emph{\DUrole{n}{radians}\DUrole{o}{=}\DUrole{default_value}{False}}, \emph{\DUrole{n}{backend}\DUrole{o}{=}\DUrole{default_value}{\textquotesingle{}FE\textquotesingle{}}}}{}
\pysigstopsignatures
\sphinxAtStartPar
Solve the system of differential equations for a given impact scenario
\begin{quote}\begin{description}
\sphinxlineitem{Parameters}\begin{itemize}
\item {} 
\sphinxAtStartPar
\sphinxstyleliteralstrong{\sphinxupquote{radius}} (\sphinxstyleliteralemphasis{\sphinxupquote{float}}) \textendash{} The radius of the asteroid in meters

\item {} 
\sphinxAtStartPar
\sphinxstyleliteralstrong{\sphinxupquote{velocity}} (\sphinxstyleliteralemphasis{\sphinxupquote{float}}) \textendash{} The entery speed of the asteroid in meters/second

\item {} 
\sphinxAtStartPar
\sphinxstyleliteralstrong{\sphinxupquote{density}} (\sphinxstyleliteralemphasis{\sphinxupquote{float}}) \textendash{} The density of the asteroid in kg/m\textasciicircum{}3

\item {} 
\sphinxAtStartPar
\sphinxstyleliteralstrong{\sphinxupquote{strength}} (\sphinxstyleliteralemphasis{\sphinxupquote{float}}) \textendash{} The strength of the asteroid (i.e. the maximum pressure it can
take before fragmenting) in N/m\textasciicircum{}2

\item {} 
\sphinxAtStartPar
\sphinxstyleliteralstrong{\sphinxupquote{angle}} (\sphinxstyleliteralemphasis{\sphinxupquote{float}}) \textendash{} The initial trajectory angle of the asteroid to the horizontal
By default, input is in degrees. If ‘radians’ is set to True, the
input should be in radians

\item {} 
\sphinxAtStartPar
\sphinxstyleliteralstrong{\sphinxupquote{init\_altitude}} (\sphinxstyleliteralemphasis{\sphinxupquote{float}}\sphinxstyleliteralemphasis{\sphinxupquote{, }}\sphinxstyleliteralemphasis{\sphinxupquote{optional}}) \textendash{} Initial altitude in m

\item {} 
\sphinxAtStartPar
\sphinxstyleliteralstrong{\sphinxupquote{dt}} (\sphinxstyleliteralemphasis{\sphinxupquote{float}}\sphinxstyleliteralemphasis{\sphinxupquote{, }}\sphinxstyleliteralemphasis{\sphinxupquote{optional}}) \textendash{} The output timestep, in s

\item {} 
\sphinxAtStartPar
\sphinxstyleliteralstrong{\sphinxupquote{radians}} (\sphinxstyleliteralemphasis{\sphinxupquote{logical}}\sphinxstyleliteralemphasis{\sphinxupquote{, }}\sphinxstyleliteralemphasis{\sphinxupquote{optional}}) \textendash{} Whether angles should be given in degrees or radians. Default=False
Angles returned in the dataframe will have the same units as the
input

\item {} 
\sphinxAtStartPar
\sphinxstyleliteralstrong{\sphinxupquote{backend}} (\sphinxstyleliteralemphasis{\sphinxupquote{str}}\sphinxstyleliteralemphasis{\sphinxupquote{, }}\sphinxstyleliteralemphasis{\sphinxupquote{optional}}) \textendash{} Which solving method to use. Default=’FE’

\end{itemize}

\sphinxlineitem{Returns}
\sphinxAtStartPar
\sphinxstylestrong{Result} \textendash{} A pandas dataframe containing the solution to the system.
Includes the following columns:
‘velocity’, ‘mass’, ‘angle’, ‘altitude’,
‘distance’, ‘radius’, ‘time’

\sphinxlineitem{Return type}
\sphinxAtStartPar
DataFrame

\end{description}\end{quote}

\end{fulllineitems}

\index{solve\_atmospheric\_entry\_FE() (solver.Planet method)@\spxentry{solve\_atmospheric\_entry\_FE()}\spxextra{solver.Planet method}}

\begin{fulllineitems}
\phantomsection\label{\detokenize{index:solver.Planet.solve_atmospheric_entry_FE}}
\pysigstartsignatures
\pysiglinewithargsret{\sphinxbfcode{\sphinxupquote{solve\_atmospheric\_entry\_FE}}}{\emph{\DUrole{n}{radius}}, \emph{\DUrole{n}{velocity}}, \emph{\DUrole{n}{angle}}, \emph{\DUrole{n}{init\_altitude}}, \emph{\DUrole{n}{dt}}}{}
\pysigstopsignatures
\sphinxAtStartPar
Solve the system of differential equations for a given impact scenario
using forward Eular method
\begin{quote}\begin{description}
\sphinxlineitem{Parameters}\begin{itemize}
\item {} 
\sphinxAtStartPar
\sphinxstyleliteralstrong{\sphinxupquote{radius}} (\sphinxstyleliteralemphasis{\sphinxupquote{float}}) \textendash{} The radius of the asteroid in meters

\item {} 
\sphinxAtStartPar
\sphinxstyleliteralstrong{\sphinxupquote{velocity}} (\sphinxstyleliteralemphasis{\sphinxupquote{float}}) \textendash{} The entery speed of the asteroid in meters/second

\item {} 
\sphinxAtStartPar
\sphinxstyleliteralstrong{\sphinxupquote{angle}} (\sphinxstyleliteralemphasis{\sphinxupquote{float}}) \textendash{} The initial trajectory angle of the asteroid to the horizontal
By default, input is in degrees. If ‘radians’ is set to True, the
input should be in radians

\item {} 
\sphinxAtStartPar
\sphinxstyleliteralstrong{\sphinxupquote{init\_altitude}} (\sphinxstyleliteralemphasis{\sphinxupquote{float}}\sphinxstyleliteralemphasis{\sphinxupquote{, }}\sphinxstyleliteralemphasis{\sphinxupquote{optional}}) \textendash{} Initial altitude in m

\item {} 
\sphinxAtStartPar
\sphinxstyleliteralstrong{\sphinxupquote{dt}} (\sphinxstyleliteralemphasis{\sphinxupquote{float}}\sphinxstyleliteralemphasis{\sphinxupquote{, }}\sphinxstyleliteralemphasis{\sphinxupquote{optional}}) \textendash{} The output timestep, in s

\end{itemize}

\sphinxlineitem{Return type}
\sphinxAtStartPar
None

\end{description}\end{quote}

\end{fulllineitems}

\index{solve\_atmospheric\_entry\_RK4() (solver.Planet method)@\spxentry{solve\_atmospheric\_entry\_RK4()}\spxextra{solver.Planet method}}

\begin{fulllineitems}
\phantomsection\label{\detokenize{index:solver.Planet.solve_atmospheric_entry_RK4}}
\pysigstartsignatures
\pysiglinewithargsret{\sphinxbfcode{\sphinxupquote{solve\_atmospheric\_entry\_RK4}}}{\emph{\DUrole{n}{radius}}, \emph{\DUrole{n}{velocity}}, \emph{\DUrole{n}{angle}}, \emph{\DUrole{n}{init\_altitude}}, \emph{\DUrole{n}{dt}}}{}
\pysigstopsignatures
\sphinxAtStartPar
Solve the system of differential equations for a given impact scenario
using RK4 method
\begin{quote}\begin{description}
\sphinxlineitem{Parameters}\begin{itemize}
\item {} 
\sphinxAtStartPar
\sphinxstyleliteralstrong{\sphinxupquote{radius}} (\sphinxstyleliteralemphasis{\sphinxupquote{float}}) \textendash{} The radius of the asteroid in meters

\item {} 
\sphinxAtStartPar
\sphinxstyleliteralstrong{\sphinxupquote{velocity}} (\sphinxstyleliteralemphasis{\sphinxupquote{float}}) \textendash{} The entery speed of the asteroid in meters/second

\item {} 
\sphinxAtStartPar
\sphinxstyleliteralstrong{\sphinxupquote{angle}} (\sphinxstyleliteralemphasis{\sphinxupquote{float}}) \textendash{} The initial trajectory angle of the asteroid to the horizontal
By default, input is in degrees. If ‘radians’ is set to True, the
input should be in radians

\item {} 
\sphinxAtStartPar
\sphinxstyleliteralstrong{\sphinxupquote{init\_altitude}} (\sphinxstyleliteralemphasis{\sphinxupquote{float}}\sphinxstyleliteralemphasis{\sphinxupquote{, }}\sphinxstyleliteralemphasis{\sphinxupquote{optional}}) \textendash{} Initial altitude in m

\item {} 
\sphinxAtStartPar
\sphinxstyleliteralstrong{\sphinxupquote{dt}} (\sphinxstyleliteralemphasis{\sphinxupquote{float}}\sphinxstyleliteralemphasis{\sphinxupquote{, }}\sphinxstyleliteralemphasis{\sphinxupquote{optional}}) \textendash{} The output timestep, in s

\end{itemize}

\sphinxlineitem{Return type}
\sphinxAtStartPar
None

\end{description}\end{quote}

\end{fulllineitems}


\end{fulllineitems}

\phantomsection\label{\detokenize{index:module-damage}}\index{module@\spxentry{module}!damage@\spxentry{damage}}\index{damage@\spxentry{damage}!module@\spxentry{module}}\index{damage\_zones() (in module damage)@\spxentry{damage\_zones()}\spxextra{in module damage}}

\begin{fulllineitems}
\phantomsection\label{\detokenize{index:damage.damage_zones}}
\pysigstartsignatures
\pysiglinewithargsret{\sphinxcode{\sphinxupquote{damage.}}\sphinxbfcode{\sphinxupquote{damage\_zones}}}{\emph{\DUrole{n}{outcome}}, \emph{\DUrole{n}{lat}}, \emph{\DUrole{n}{lon}}, \emph{\DUrole{n}{bearing}}, \emph{\DUrole{n}{pressures}}, \emph{\DUrole{n}{map}\DUrole{o}{=}\DUrole{default_value}{False}}}{}
\pysigstopsignatures
\sphinxAtStartPar
Calculate the latitude and longitude of the surface zero location and the
list of airblast damage radii (m) for a given impact scenario.
\begin{quote}\begin{description}
\sphinxlineitem{Parameters}\begin{itemize}
\item {} 
\sphinxAtStartPar
\sphinxstyleliteralstrong{\sphinxupquote{outcome}} (\sphinxstyleliteralemphasis{\sphinxupquote{Dict}}) \textendash{} the outcome dictionary from an impact scenario

\item {} 
\sphinxAtStartPar
\sphinxstyleliteralstrong{\sphinxupquote{lat}} (\sphinxstyleliteralemphasis{\sphinxupquote{float}}) \textendash{} latitude of the meteoroid entry point (degrees)

\item {} 
\sphinxAtStartPar
\sphinxstyleliteralstrong{\sphinxupquote{lon}} (\sphinxstyleliteralemphasis{\sphinxupquote{float}}) \textendash{} longitude of the meteoroid entry point (degrees)

\item {} 
\sphinxAtStartPar
\sphinxstyleliteralstrong{\sphinxupquote{bearing}} (\sphinxstyleliteralemphasis{\sphinxupquote{float}}) \textendash{} Bearing (azimuth) relative to north of meteoroid trajectory (degrees)

\item {} 
\sphinxAtStartPar
\sphinxstyleliteralstrong{\sphinxupquote{pressures}} (\sphinxstyleliteralemphasis{\sphinxupquote{float}}\sphinxstyleliteralemphasis{\sphinxupquote{, }}\sphinxstyleliteralemphasis{\sphinxupquote{arraylike}}) \textendash{} List of threshold pressures to define airblast damage levels

\item {} 
\sphinxAtStartPar
\sphinxstyleliteralstrong{\sphinxupquote{plot}} (\sphinxstyleliteralemphasis{\sphinxupquote{bool}}) \textendash{} Boolean value to decide plotting

\end{itemize}

\sphinxlineitem{Returns}
\sphinxAtStartPar
\begin{itemize}
\item {} 
\sphinxAtStartPar
\sphinxstylestrong{blat} (\sphinxstyleemphasis{float}) \textendash{} latitude of the surface zero point (degrees)

\item {} 
\sphinxAtStartPar
\sphinxstylestrong{blon} (\sphinxstyleemphasis{float}) \textendash{} longitude of the surface zero point (degrees)

\item {} 
\sphinxAtStartPar
\sphinxstylestrong{damrad} (\sphinxstyleemphasis{arraylike, float}) \textendash{} List of distances specifying the blast radii
for the input damage levels

\item {} 
\sphinxAtStartPar
\sphinxstylestrong{plot} (\sphinxstyleemphasis{plot object}) \textendash{} The plot specifying the areas effected by
each damage level

\end{itemize}


\end{description}\end{quote}
\subsubsection*{Examples}

\begin{sphinxVerbatim}[commandchars=\\\{\}]
\PYG{g+gp}{\PYGZgt{}\PYGZgt{}\PYGZgt{} }\PYG{k+kn}{import} \PYG{n+nn}{armageddon}
\PYG{g+gp}{\PYGZgt{}\PYGZgt{}\PYGZgt{} }\PYG{n}{outcome} \PYG{o}{=} \PYG{p}{\PYGZob{}}\PYG{l+s+s1}{\PYGZsq{}}\PYG{l+s+s1}{burst\PYGZus{}altitude}\PYG{l+s+s1}{\PYGZsq{}}\PYG{p}{:} \PYG{l+m+mf}{8e3}\PYG{p}{,} \PYG{l+s+s1}{\PYGZsq{}}\PYG{l+s+s1}{burst\PYGZus{}energy}\PYG{l+s+s1}{\PYGZsq{}}\PYG{p}{:} \PYG{l+m+mf}{7e3}\PYG{p}{,}
\PYG{g+go}{               \PYGZsq{}burst\PYGZus{}distance\PYGZsq{}: 90e3, \PYGZsq{}burst\PYGZus{}peak\PYGZus{}dedz\PYGZsq{}: 1e3,}
\PYG{g+go}{               \PYGZsq{}outcome\PYGZsq{}: \PYGZsq{}Airburst\PYGZsq{}\PYGZcb{}}
\PYG{g+gp}{\PYGZgt{}\PYGZgt{}\PYGZgt{} }\PYG{n}{armageddon}\PYG{o}{.}\PYG{n}{damage\PYGZus{}zones}\PYG{p}{(}\PYG{n}{outcome}\PYG{p}{,} \PYG{l+m+mf}{52.79}\PYG{p}{,} \PYG{o}{\PYGZhy{}}\PYG{l+m+mf}{2.95}\PYG{p}{,} \PYG{l+m+mi}{135}\PYG{p}{,}
\PYG{g+go}{                            pressures=[1e3, 3.5e3, 27e3, 43e3])}
\end{sphinxVerbatim}

\end{fulllineitems}

\index{impact\_risk() (in module damage)@\spxentry{impact\_risk()}\spxextra{in module damage}}

\begin{fulllineitems}
\phantomsection\label{\detokenize{index:damage.impact_risk}}
\pysigstartsignatures
\pysiglinewithargsret{\sphinxcode{\sphinxupquote{damage.}}\sphinxbfcode{\sphinxupquote{impact\_risk}}}{\emph{\DUrole{n}{planet}}, \emph{\DUrole{n}{means}\DUrole{o}{=}\DUrole{default_value}{\{\textquotesingle{}angle\textquotesingle{}: 45, \textquotesingle{}bearing\textquotesingle{}: 115.0, \textquotesingle{}density\textquotesingle{}: 3000, \textquotesingle{}lat\textquotesingle{}: 53.0, \textquotesingle{}lon\textquotesingle{}: \sphinxhyphen{}2.5, \textquotesingle{}radius\textquotesingle{}: 35, \textquotesingle{}strength\textquotesingle{}: 10000000.0, \textquotesingle{}velocity\textquotesingle{}: 19000.0\}}}, \emph{\DUrole{n}{stdevs}\DUrole{o}{=}\DUrole{default_value}{\{\textquotesingle{}angle\textquotesingle{}: 1, \textquotesingle{}bearing\textquotesingle{}: 0.5, \textquotesingle{}density\textquotesingle{}: 500, \textquotesingle{}lat\textquotesingle{}: 0.025, \textquotesingle{}lon\textquotesingle{}: 0.025, \textquotesingle{}radius\textquotesingle{}: 1, \textquotesingle{}strength\textquotesingle{}: 5000000.0, \textquotesingle{}velocity\textquotesingle{}: 1000.0\}}}, \emph{\DUrole{n}{pressure}\DUrole{o}{=}\DUrole{default_value}{27000.0}}, \emph{\DUrole{n}{nsamples}\DUrole{o}{=}\DUrole{default_value}{10}}, \emph{\DUrole{n}{sector}\DUrole{o}{=}\DUrole{default_value}{True}}}{}
\pysigstopsignatures
\sphinxAtStartPar
Perform an uncertainty analysis to calculate the risk for each affected
UK postcode or postcode sector
\begin{quote}\begin{description}
\sphinxlineitem{Parameters}\begin{itemize}
\item {} 
\sphinxAtStartPar
\sphinxstyleliteralstrong{\sphinxupquote{planet}} (\sphinxstyleliteralemphasis{\sphinxupquote{armageddon.Planet instance}}) \textendash{} The Planet instance from which to solve the atmospheric entry

\item {} 
\sphinxAtStartPar
\sphinxstyleliteralstrong{\sphinxupquote{means}} (\sphinxstyleliteralemphasis{\sphinxupquote{dict}}) \textendash{} A dictionary of mean input values for the uncertainty analysis. This
should include values for \sphinxcode{\sphinxupquote{radius}}, \sphinxcode{\sphinxupquote{angle}}, \sphinxcode{\sphinxupquote{strength}},
\sphinxcode{\sphinxupquote{density}}, \sphinxcode{\sphinxupquote{velocity}}, \sphinxcode{\sphinxupquote{lat}}, \sphinxcode{\sphinxupquote{lon}} and \sphinxcode{\sphinxupquote{bearing}}

\item {} 
\sphinxAtStartPar
\sphinxstyleliteralstrong{\sphinxupquote{stdevs}} (\sphinxstyleliteralemphasis{\sphinxupquote{dict}}) \textendash{} A dictionary of standard deviations for each input value. This
should include values for \sphinxcode{\sphinxupquote{radius}}, \sphinxcode{\sphinxupquote{angle}}, \sphinxcode{\sphinxupquote{strength}},
\sphinxcode{\sphinxupquote{density}}, \sphinxcode{\sphinxupquote{velocity}}, \sphinxcode{\sphinxupquote{lat}}, \sphinxcode{\sphinxupquote{lon}} and \sphinxcode{\sphinxupquote{bearing}}

\item {} 
\sphinxAtStartPar
\sphinxstyleliteralstrong{\sphinxupquote{pressure}} (\sphinxstyleliteralemphasis{\sphinxupquote{float}}) \textendash{} A single pressure at which to calculate the damage zone for each impact

\item {} 
\sphinxAtStartPar
\sphinxstyleliteralstrong{\sphinxupquote{nsamples}} (\sphinxstyleliteralemphasis{\sphinxupquote{int}}) \textendash{} The number of iterations to perform in the uncertainty analysis

\item {} 
\sphinxAtStartPar
\sphinxstyleliteralstrong{\sphinxupquote{sector}} (\sphinxstyleliteralemphasis{\sphinxupquote{logical}}\sphinxstyleliteralemphasis{\sphinxupquote{, }}\sphinxstyleliteralemphasis{\sphinxupquote{optional}}) \textendash{} If True (default) calculate the risk for postcode sectors, otherwise
calculate the risk for postcodes

\end{itemize}

\sphinxlineitem{Returns}
\sphinxAtStartPar
\sphinxstylestrong{risk} \textendash{} A pandas DataFrame with columns for postcode (or postcode sector) and
the associated risk. These should be called \sphinxcode{\sphinxupquote{postcode}} or \sphinxcode{\sphinxupquote{sector}},
and \sphinxcode{\sphinxupquote{risk}}.

\sphinxlineitem{Return type}
\sphinxAtStartPar
DataFrame

\end{description}\end{quote}

\end{fulllineitems}

\phantomsection\label{\detokenize{index:module-mapping}}\index{module@\spxentry{module}!mapping@\spxentry{mapping}}\index{mapping@\spxentry{mapping}!module@\spxentry{module}}\index{plot\_circle() (in module mapping)@\spxentry{plot\_circle()}\spxextra{in module mapping}}

\begin{fulllineitems}
\phantomsection\label{\detokenize{index:mapping.plot_circle}}
\pysigstartsignatures
\pysiglinewithargsret{\sphinxcode{\sphinxupquote{mapping.}}\sphinxbfcode{\sphinxupquote{plot\_circle}}}{\emph{\DUrole{n}{lat}}, \emph{\DUrole{n}{lon}}, \emph{\DUrole{n}{radius}}, \emph{\DUrole{n}{map}\DUrole{o}{=}\DUrole{default_value}{None}}, \emph{\DUrole{o}{**}\DUrole{n}{kwargs}}}{}
\pysigstopsignatures
\sphinxAtStartPar
Plot a circle on a map (creating a new folium map instance if necessary).
\begin{quote}\begin{description}
\sphinxlineitem{Parameters}\begin{itemize}
\item {} 
\sphinxAtStartPar
\sphinxstyleliteralstrong{\sphinxupquote{lat}} (\sphinxstyleliteralemphasis{\sphinxupquote{float}}) \textendash{} latitude of circle to plot (degrees)

\item {} 
\sphinxAtStartPar
\sphinxstyleliteralstrong{\sphinxupquote{lon}} (\sphinxstyleliteralemphasis{\sphinxupquote{float}}) \textendash{} longitude of circle to plot (degrees)

\item {} 
\sphinxAtStartPar
\sphinxstyleliteralstrong{\sphinxupquote{radius}} (\sphinxstyleliteralemphasis{\sphinxupquote{float}}) \textendash{} radius of circle to plot (m)

\item {} 
\sphinxAtStartPar
\sphinxstyleliteralstrong{\sphinxupquote{map}} (\sphinxstyleliteralemphasis{\sphinxupquote{folium.Map}}) \textendash{} existing map object

\end{itemize}

\sphinxlineitem{Return type}
\sphinxAtStartPar
Folium map object

\end{description}\end{quote}
\subsubsection*{Examples}

\begin{sphinxVerbatim}[commandchars=\\\{\}]
\PYG{g+gp}{\PYGZgt{}\PYGZgt{}\PYGZgt{} }\PYG{k+kn}{import} \PYG{n+nn}{folium}
\PYG{g+gp}{\PYGZgt{}\PYGZgt{}\PYGZgt{} }\PYG{n}{armageddon}\PYG{o}{.}\PYG{n}{plot\PYGZus{}circle}\PYG{p}{(}\PYG{l+m+mf}{52.79}\PYG{p}{,} \PYG{o}{\PYGZhy{}}\PYG{l+m+mf}{2.95}\PYG{p}{,} \PYG{l+m+mf}{1e3}\PYG{p}{,} \PYG{n+nb}{map}\PYG{o}{=}\PYG{k+kc}{None}\PYG{p}{)}
\end{sphinxVerbatim}

\end{fulllineitems}



\renewcommand{\indexname}{Python Module Index}
\begin{sphinxtheindex}
\let\bigletter\sphinxstyleindexlettergroup
\bigletter{d}
\item\relax\sphinxstyleindexentry{damage}\sphinxstyleindexpageref{index:\detokenize{module-damage}}
\indexspace
\bigletter{l}
\item\relax\sphinxstyleindexentry{locator}\sphinxstyleindexpageref{index:\detokenize{module-locator}}
\indexspace
\bigletter{m}
\item\relax\sphinxstyleindexentry{mapping}\sphinxstyleindexpageref{index:\detokenize{module-mapping}}
\indexspace
\bigletter{s}
\item\relax\sphinxstyleindexentry{solver}\sphinxstyleindexpageref{index:\detokenize{module-solver}}
\end{sphinxtheindex}

\renewcommand{\indexname}{Index}
\printindex
\end{document}